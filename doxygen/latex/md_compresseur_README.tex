Le contrôle continu est soumis à la règle du max , il n’est donc pas indispensable mais est fortement recommandé !

Les démos sur machine du projet « Code de Huffman » auront lieu fin décembre. Si vous désirez y participer, le binôme (groupe de deux étudiants) doit \+:
\begin{DoxyItemize}
\item envoyer à Michel Meynard (\href{mailto:meynard@lirmm.fr}{\tt meynard@lirmm.\+fr}) et Pierre Pompidor (\href{mailto:pompidor@lirmm.fr}{\tt pompidor@lirmm.\+fr}), une archive de vos codes 2 J\+O\+U\+RS A\+V\+A\+NT LA S\+O\+U\+T\+E\+N\+A\+N\+CE ;
\item en retour, nous vous enverrons un créneau plus précis de votre passage.
\end{DoxyItemize}

Veuillez dans votre archive (nommée par les noms des participants au projet), créer un petit fichier R\+E\+A\+D\+ME qui indiquera l’état d’avancement de votre projet (et par exemple les bogues résiduels). Par ailleurs, T\+O\+US les participants au projet devront être présents (les absents ne seront sinon pas notés).

Le jour de la soutenance, les participants doivent venir avec \+:
\begin{DoxyItemize}
\item un micro rapport papier de 1 à 2 pages décrivant les choix importants effectués ;
\item les listings (fichiers sources) documentés (doxygen) ;
\item des réponses aux questions indiquées dans le CC ;
\end{DoxyItemize}

Chaque soutenance durera 12 à 15 minutes ! C’est extrêmement rapide, aussi soyez certain d’avoir les exécutables prêts à fonctionner sur votre compte Unix ainsi que les fichiers exemples. Une recompilation du projet durant la soutenance n’est pas envisageable !

une démonstration de l’application durant 5 minutes (utilisant des fichiers fournis par les examinateurs) ; — des réponses précises des 2 étudiants aux questions posées durant 5 minutes ; — un micro rapport papier de 1 à 2 pages décrivant les choix importants effectués ; — des listings papiers commentés (doxygen)

Quel est le nombre maximum de caractères (char) différents ? — Comment représenter l’arbre de Huffman ? Si l’arbre est implémenté avec des tableaux (fg, fd, parent), quels sont les indices des feuilles ? Quelle est la taille maximale de l’arbre (nombre de noeuds) ? — Comment les caractères présents sont-\/ils codés dans l’arbre ? — Le préfixe du fichier compressé doit-\/il nécessairement contenir l’arbre ou les codes des caractères ou bien les deux (critère d’efficacité) ? — Quelle est la taille minimale de ce préfixe (expliquer chaque champ et sa longueur) ? — Si le dernier caractère écrit ne finit pas sur une frontière d’octet, comment le compléter ? Comment ne pas prendre les bits de complétion pour des bits de données ? — Le décompresseur doit-\/il reconstituer l’arbre ? Comment ?

Le projet est décomposé en deux programmes C \+: huf.\+c le programme de compression utilisé selon la syntaxe suivante \+: \$huf source dest où source est un fichier quelconque et où dest est le nom du fichier généré par compression à la Huffman du fichier source . Cette compression doit afficher les informations suivantes sur la sortie standard \+: — liste des caractères et de leur probabilité d’apparition ; — arbre de Huffman (tableaux fg fd parent ou par indentation) ; — affichage des codes de chaque caractère (code\+Char(\+E)=010100) ; — longueur moyenne de codage ; — taille originelle de la source, taille compressée et gain en pourcentage comme dans l’exemple suivant \+: Taille originelle \+: 5194; taille compressée \+: 3761; gain \+: 27.\+6\% ! dehuf.\+c le programme de décompression utilisé selon la syntaxe suivante \+: \$dehuf dest où dest est un fichier compressé à la Huffman. Le fichier décompressé sera envoyé directement sur la sortie standard. 