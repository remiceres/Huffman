\subsection*{A faire}


\begin{DoxyItemize}
\item un micro rapport papier de 1 à 2 pages décrivant les choix importants effectués P\+A\+P\+I\+ER;
\item Taille originelle de la source, taille compressée et gain en pourcentage (Taille originelle \+: 5194; taille compressée \+: 3761; gain \+: 27.\+6\% !)
\item Fichier R\+E\+A\+D\+ME qui indiquera l’état d’avancement de votre projet (et par exemple les bogues résiduels)
\item les listings (fichiers sources) documentés (doxygen) P\+A\+P\+I\+ER
\item doxygen
\item cd/latex
\item make
\item Archive nommée par les noms du groupe
\item Envoyer à \href{mailto:meynard@lirmm.fr}{\tt meynard@lirmm.\+fr} et \href{mailto:pompidor@lirmm.fr}{\tt pompidor@lirmm.\+fr}
\end{DoxyItemize}






\begin{DoxyItemize}
\item Quel est le nombre maximum de caractères (char) différents ?
\begin{DoxyItemize}
\item Le nombre maximum de caractères est 256.
\end{DoxyItemize}
\item Comment représenter l’arbre de Huffman ? Si l’arbre est implémenté avec des tableaux (fg, fd, parent), quels sont les indices des feuilles ? Quelle est la taille maximale de l’arbre (nombre de noeuds) ?
\begin{DoxyItemize}
\item L\textquotesingle{}arbre de huffman est représenté par une structure possédant les varriables {\ttfamily pere},{\ttfamily fg}, {\ttfamily fd} et {\ttfamily frequences}.
\item Si l’arbre est implémenté avec des tableaux (fg, fd, parent), les indices des feuilles corresponde au code du carractère.
\item L\textquotesingle{}arbre peut avoir au maximum 256 Noeuds.
\end{DoxyItemize}
\item Comment les caractères présents sont-\/ils codés dans l’arbre ?
\begin{DoxyItemize}
\item il sont codé par leurs code A\+S\+Cii.
\end{DoxyItemize}
\item Le préfixe du fichier compressé doit-\/il nécessairement contenir l’arbre ou les codes des caractères ou bien les deux (critère d’efficacité) ?
\begin{DoxyItemize}
\item Le préfixe du fichier compressé doit contenir soit l\textquotesingle{}arbre soit les codes des caractères.
\item Stocker l\textquotesingle{}arbre est plus efficace que stoker les codes des caractères.
\end{DoxyItemize}
\item Quelle est la taille minimale de ce préfixe (expliquer chaque champ et sa longueur) ?
\begin{DoxyItemize}
\item Le préfixe est composé de 3 octet magiques permétant d\textquotesingle{}identifié pouvant être décompréssé. 1 octet représentant le nombre de bits utile du dernier octet puis l\textquotesingle{}arbre.
\end{DoxyItemize}
\item Si le dernier caractère écrit ne finit pas sur une frontière d’octet, comment le compléter ? Comment ne pas prendre les bits de complétion pour des bits de données ?
\begin{DoxyItemize}
\item on le complete avc des 0 inutile.
\item Lors de la décomprétion on vérifie si on est a la fin du fichier, si c\textquotesingle{}est le cas on ne traite que les bits utile récupéré dans l\textquotesingle{}entête
\end{DoxyItemize}
\item Le décompresseur doit-\/il reconstituer l’arbre ? Comment ?
\begin{DoxyItemize}
\item oui, a partir de l\textquotesingle{}entête. 
\end{DoxyItemize}
\end{DoxyItemize}